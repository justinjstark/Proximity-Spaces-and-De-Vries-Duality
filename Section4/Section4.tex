\section{De Vries Algebras}
\label{devries}

\begin{definition}
	A \defn{de Vries algebra} is a Boolean algebra \( (B,\leq) \) with a binary relation \( \sdev \) on \( B \) such that for all \( a,b,c,d \in B \),
	\begin{enumerate}
		\item[(D1)] \( 0 \dev 0 \)
		\item[(D2)] \( a \dev b \implies a \leq b \)
		\item[(D3)] \( a \dev b \implies -b \dev -a \)
		\item[(D4)] \( a \dev b \wedge c \Leftrightarrow a \dev b \textand a \dev c \)
		\item[(D5)] \( a \dev b \implies \) \( \exists c \in B \textst a \dev c \dev b \)
		\item[(D6)] \( 0 \neq a \in B \implies \exists b \in B \textst b \neq 0 \textand b \dev a \)
	\end{enumerate}
	
	Henceforth, when we say \( B \) is a de Vries algebra, we will assume \( \leq \) is the partial order of the underlying Boolean algebra and \( \dev \) is the endowed de Vries algebra relation on \( B \).
\end{definition}

\begin{proposition}
	\label{devveeunder}
	Let \( B \) be a de Vries algebra.  For all \( a \in B \),
	\[ a = \bigvee_{b \dev a} b \]
\end{proposition}
\begin{proof}
	By (D2), \( b \dev a \) implies \( b \leq a \).  It is immediate that \( \bigvee_{b \dev a} b \leq a \).
	
	Suppose, by way of contradiction, that \( \bigvee_{b \dev a} b < a \).  Then \( a - \bigvee_{b \dev a} b > 0 \).  By (D6), there is a \( c \in B \) such that \( 0 < c \dev \bigvee_{b \dev a} b \).  So \( c \leq \bigvee_{b \dev a} b \).  Also, \( c \leq a - \bigvee_{b \dev a} b \).  Therefore,
	\[ c \leq \bigvee_{b \dev a} b \wedge (a - \bigvee_{b \dev a} b) = 0 \]
	This is a contradiction.
\end{proof}

\begin{example}
	Let \( (B,\leq) \) be a complete Boolean algebra.  Then \( (B,\leq) \) is a de Vries algebra.
\end{example}

\begin{theorem}%[MacNeille \cite{macneille} and Tarski \cite{tarski}]
	Let \( (X, \tau) \) be a space.  Then \( (\RO(X),\leq) \) is a complete Boolean algebra with the following operations.  For \( U,V \in \RO(X) \),
	\begin{itemize}
		\item \( 1 = X \)
		\item \( U \wedge V = U \cap V \)
		\item \( U \vee V = \int\cl(U \cup V) \)
		\item \( -U = X \setminus \cl{U} \)
	\end{itemize}
	If \( \{U_i\}_{i \in A} \) is an infinite collection of regular open sets of \( X \), then
	\begin{itemize}
		\item \( \bigwedge_{i \in A} U_i = \int{\bigcap_{i \in A} U_i} \)
		\item \( \bigvee_{i \in A} U_i = \int\cl{\bigcup_{i \in A} U_i} \)
	\end{itemize}
	The partial order \( \leq \) on \( (\RO(X),\leq) \) is set inclusion.
\end{theorem}
\begin{proof}
	\cite[7.21 \& 7.22]{polkowski}.
\end{proof}

\begin{example}
	\label{devexro}
	Let \( (X,\tau) \) be a compact Hausdorff space.  For \( U,V \in \RO(X) \), define \( U \dev V \) if and only if \( \cl{U} \subseteq V \).  Then \( (\RO(X),\sdev) \) is a de Vries algebra.
\end{example}

\begin{example}
	\label{PXdev}
	Let \( (X,\ll) \) be a proximity space.  Then \( (\cP(X),\ll) \) is a de Vries algebra if and only if \( (X,\tau(\ll)) \) is discrete.
\end{example}
\begin{proof}
	It is well known that \( (\cP(X),\ll) \) is a complete Boolean algebra with the partial order as subset inclusion.

	Whether \( (X,\tau(\ll)) \) is discrete or not, (D1) through (D5) are clearly satisfied.  It is enough to check (D6).

	Suppose \( X \) is discrete and \( A \subseteq X \) such that \( A \neq \emptyset \).  Then there is an \( a \in A \).  Since \( A \in \tau(\ll) \) then \( a \nnear X \setminus A \).  Hence \( \{ x \} \ll A \) and thus (D6) is satisfied.
	
	Conversely, suppose \( X \) is not discrete.  Then there is a limit point \( x \) in \( X \).  Note that \( \{x\} \) is not open.  So \( X \setminus \{x\} \) is not closed.  Hence it must be that \( x \near X \setminus \{x\} \).  Thus \( \{x\} \nll \{x\} \).  So if there is a set \( A \subseteq X \) such that \( A \ll \{x\} \) then \( A = \emptyset \).  Therefore, (D6) is not satisfied.
\end{proof}

\begin{example}
	Let \( (X,\ll) \) be a proximity space.  Then \( (\RO(X),\ll) \) is a de Vries algebra.
\end{example}
\begin{proof}
	It is enough to check (D6).  Let \( \emptyset \neq U \in \RO(X) \).  Take \( x \in U \).  Since \( U \) is open, \( x \ll U \).  By \ref{llinsertro}, there is a \( V \in \RO(X) \) such that \( x \ll V \ll U \).
\end{proof}

\begin{remark}
	By \ref{PXdev}, it is clear that a de Vries algebra and a proximity space \( (X,\sll) \) are similar structures.  As a result, they have many of the same properties.
\end{remark}

\begin{proposition}
	\label{devextend}
	Let \( B \) be a de Vries algebra.  For all \( a,b,c,d \in B \),
	\[ a \leq b \dev c \leq d \implies a \dev d \]
\end{proposition}
\begin{proof}
	The proof is analogous to \ref{llextend}.
\end{proof}

\begin{proposition}
	\label{devwedgevee}
	Let \( B \) be a de Vries algebra and \( \{ a_i \}_{i=1}^n \), \( \{ b_i \}_{i=1}^n \) finite families of elements of \( B \) such that \( a_i \dev b_i \) for \( i=1, \dots, n \).
	\begin{enumerate}[label={(\arabic*)},ref={\theproposition(\arabic*)}]
		\item \label{devwedge}
			\( \bigwedge_{i=1}^n a_i \dev \bigwedge_{i=1}^n b_i \)
		\item \label{devvee}
			\( \bigvee_{i=1}^n a_i \dev \bigvee_{i=1}^n b_i \)
	\end{enumerate}
\end{proposition}
\begin{proof}
	The proof is analogous to \ref{llintun}.
\end{proof}

\begin{definition}
	Let \( B \) be a de Vries algebra.
	\begin{itemize}
		\item A \defn{round filter} on \( B \) is a filter \( \cF \) on \( B \) such that for all \( a \in \cF \), there is a \( b \in \cF \) such that \( b \dev a \).
		\item A maximal round filter on \( B \) is called an \defn{end}.  We denote the set of all ends on \( B \) as \( \End(B) \).
		\item The \defn{round hull} of a filter \( \cF \) on \( B \) is \( \rnd(\cF) = \{ a \in B: \exists b \in \cF, b \dev a \} \).
	\end{itemize}
\end{definition}

\begin{proposition}
	Let \( B \) be a de Vries algebra.
	\begin{enumerate}[label={(\arabic*)},ref={\theproposition(\arabic*)}]
		\item Every round filter on \( B \) is contained in an end.
		\item A round filter \( \cF \) on \( B \) is an end if and only if for all \( a,b \in B \), \( a \dev b \) implies \( -a \in \cF \) or \( b \in \cF \).
		\item For a filter \( \cF \) on \( B \), \( \rnd(\cF) \) is a round filter and \( \rnd(\cF) \subseteq \cF \).
	\end{enumerate}
\end{proposition}
\begin{proof}
	The proofs are analogous to \ref{roundinend}, \ref{endllin}, and \ref{rndround} respectively.
\end{proof}

\begin{definition}
	For \( a \in B \), define \( \o(a) = \{ \cE \in \End(B): a \in \cE \} \)
\end{definition}

\begin{proposition}
	Let \( B \) be a de Vries algebra.  For all \( a,b \in B \),
	\begin{enumerate}[label={(\arabic*)},ref={\theproposition(\arabic*)}]
		\item \label{devoempty}
			\( a = 0 \iff \o(a) = \emptyset \)
		\item \label{devosubset}
			\( a \leq b \implies \o(a) \subseteq \o(b) \)
		\item \label{devoint}
			\( \o(a) \cap \o(b) = \o(a \wedge b) \)
		\item \label{devoun}
			\( \o(a) \cup \o(b) \subseteq \o(a \vee b) \)
		\item \label{devo}
			\( a \dev b \implies \End(B) \setminus \o(b) \subseteq \o(-a) \)
		\item \label{devobase}
			\( \{ \o(a): a \in B \} \) is a base for a compact Hausdorff topology on \( \End(B) \).
		\item \label{devocomp}
			\( \o(-a) = \End(B) \setminus \cl(\o(a)) \)
	\end{enumerate}
\end{proposition}
\begin{proof}
	The proofs are analogous to \ref{oempty}; \ref{osubset}; \ref{oun}; \ref{llo}; \ref{Endbase} and \ref{endcptt}; and \ref{ocomp} respectively.
\end{proof}

%%%%%De Vries Morphisms

\begin{definition}
	Let \( A,B \) be de Vries algebras.  A function \( f:A \to B \) is a \defn{de Vries morphism} if for all \( a,b \in A \),
	\begin{enumerate}
		\item[(M1)] \( f(0) = 0 \)
		\item[(M2)] \( f(a \wedge b) = f(a) \wedge f(b) \)
		\item[(M3)] \( a \dev b \implies -f(-a) \dev f(b) \)
		\item[(M4)] \( f(a) = \bigvee_{b \dev a} f(b) \)
	\end{enumerate}
\end{definition}

\begin{proposition}
	Let \( A,B \) be de Vries algebras and \( f:A \to B \) a de Vries morphism.  For all \( a,b \in A \),
	\begin{enumerate}[label={(\arabic*)},ref={\theproposition(\arabic*)}]
		\item \label{m11}
			\( f(1) = 1 \)
		\item \label{minverse}
			\( f(a) \leq -f(-a) \)
		\item \label{mdev}
			\( a \dev b \implies f(a) \dev f(b) \)
		\item \label{mleq}
			\( a \leq b \implies f(a) \leq f(b) \)
	\end{enumerate}
\end{proposition}
\begin{proof}
	\leavevmode
	\begin{enumerate}
		\item
			By (D1), \( 0 \dev 0 \).  By (M3), \( -f(-0) \dev f(0) \).  So \( -f(1) \dev f(0) \).  By (M1), \( -f(1) \dev 0 \) and by (D3), \( -0 \dev f(1) \).  Hence \( 1 \dev f(1) \).  By (D2), \( 1 \leq f(1) \) and thus \( f(1) = 1 \).
		\item
			Note that \( 0 = f(0) = f(a \wedge -a) \).  By (M2), \( f(a \wedge -a) = f(a) \wedge f(-a) \).  Hence \( f(a) \leq -f(-a) \).
		\item
			Suppose \( a \dev b \).  By (M1), \( f(0) = 0 \).  But \( f(0) = f(a \wedge -a) = f(a) \wedge f(-a) \) by (M2).  Therefore, \( f(a) \leq -f(-a) \).  Since \( a \dev b \), by (M3), \( -f(-a) \dev f(b) \).  So we have \( f(a) \leq -f(-a) \dev f(b) \).  By \ref{devextend}, \( f(a) \dev f(b) \).
		\item
			Suppose \( a \leq b \).  Then \( a \wedge b = a \).  So \( f(a \wedge b) = f(a) \).  By (M2), \( f(a) \wedge f(b) = f(a) \).  So \( f(a) \leq f(b) \).
	\end{enumerate}
\end{proof}

\begin{remark}
	Let \( (A,\dev) \), \( (B,\dev) \) be de Vries algebras and \( f:A \to B \) a de Vries morphism.  In general, \( f(-a) \neq -f(a) \).  Therefore, \( f \) may not be a Boolean algebra homomorphism.
\end{remark}

\begin{remark}
	Let \( A,B,C \) be de Vries algebras and \( f:A \to B \), \( g:B \to C \) de Vries morphisms.  If we consider morphism composition as normal function composition, it is clear that \( g \circ f \) satisfies \( (M1) \) through \( (M3) \).  It is not necessarily true, however, that \( g \circ f \) satisfies \( (M4) \).
\end{remark}

\begin{definition}
	\label{mast}
	Let \( A,B,C \) be de Vries algebras and \( f:A \to B \), \( g:B \to C \) de Vries morphisms.  Define
	\[ (g \ast f)(a) = \bigvee_{b \dev a} (g \circ f)(b) \]
\end{definition}

\begin{proposition}
	\label{mcompmorphism}
	Let \( A,B,C \) be de Vries algebras and \( f:A \to B \), \( g:B \to C \) de Vries morphisms.  Then \( g \ast f: A \to C \) is a de Vries morphism.
\end{proposition}
\begin{proof}
	We will verify the de Vries morphism axioms.
	
	\prooflabel{(M1)}
		Note that \( (g \ast f)(0) = \bigvee_{a \dev 0} (g \circ f)(a) \).  But \( a \dev 0 \) implies \( a = 0 \).  So \( (g \ast f)(0) = (g \circ f)(0) \) and \( (g \circ f)(0) = g(f(0)) = g(0) = 0 \) by (M1).
	
	\prooflabel{(M2)}
		Let \( a,b \in A \).  Then \( (g \ast f)(a \wedge b) = \bigvee_{c \dev a \wedge b} (g \circ f)(c) \).  By (D4), \( c \dev a \wedge b \) if and only if \( c \dev a \) and \( c \dev b \).  So
		\begin{align*}
			\bigvee_{c \dev a \wedge b} (g \circ f)(c) &= \bigvee_{c \dev a} (g \circ f)(c) \wedge \bigvee_{c \dev b} (g \circ f)(c) \\
			&= (g \ast f)(a) \wedge (g \ast f)(b)
		\end{align*}
		
		Therefore, \( (g \ast f)(a \wedge b) = (g \ast f)(a) \wedge (g \ast f)(b) \).
	
	\prooflabel{(M3)}
		Suppose \( a \dev b \).  By (D5), there exist \( c,d \in A \) such that \( a \dev c \dev d \dev b \).  By (D3), \( -d \dev -c \dev -a \).  By \ref{mdev}, \( (g \circ f)(-d) \dev (g \circ f)(-c) \).  Since \( -c \dev -a \), \( (g \circ f)(-d) \dev \bigvee_{e \dev -a} (g \circ f)(e) \).  By (D3), \( -\bigvee_{e \dev -a} (g \circ f)(e) \dev (g \circ f)(d) \).  Since \( d \dev b \), \( -\bigvee_{e \dev -a} (g \circ f)(e) \dev \bigvee_{e \dev b} (g \circ f)(e) \).  Hence \( -(g \ast f)(-a) \dev (g \ast f)(b) \).
		
	\prooflabel{(M4)}
		Let \( a \in A \).  By (D5), for all \( b \in A \) such that \( b \dev a \), there is a \( c \in A \) such that \( b \dev c \dev a \).  Therefore, \( \bigvee_{b \dev a} (g \circ f)(b) \leq \bigvee_{c \dev a} \bigvee_{b \dev c} (g \circ f)(b) \).  Hence \( (g \ast f)(a) \leq \bigvee_{c \dev a} (g \ast f)(c) \).
		
		For inequality in the other direction, note that if \( b \dev a \) and \( c \dev b \) then \( c \dev a \).  Therefore, \( \bigvee_{c \dev b} (g \circ f)(c) \leq \bigvee_{c \dev a} (g \circ f)(c) \).  Hence \( (g \ast f)(b) \leq (g \ast f)(a) \) for all \( b \dev a \).  So \( \bigvee_{b \dev a} (g \ast f)(b) \leq (g \ast f)(a) \).
\end{proof}

\begin{proposition}
	Let \( A,B,C,D \) be de Vries algebras and \( f:A \to B \), \( g:B \to C \), \( h:C \to D \) de Vries morphisms.  For \( a,b \in A \),
	\begin{enumerate}[label={(\arabic*)},ref={\theproposition(\arabic*)}]
		\item \label{mastleqcirc}
			\( (g \ast f)(a) \leq (g \circ f)(a) \)
		\item \label{mdevcircleqast}
			\( a \dev b \implies (g \circ f)(a) \leq (g \ast f)(b) \)
		\item \label{mdevcompflip}
			\( a \dev b \implies ((h \ast g) \circ f)(a) \leq (h \circ (g \ast f))(b) \)
		\item \label{mdevcircleqastcomp}
			\( a \dev b \implies (h \circ (g \ast f))(a) \leq ((h \ast g) \ast f)(b) \)
	\end{enumerate}
\end{proposition}
\begin{proof}
	\leavevmode
	\begin{enumerate}		
		\item
			If \( a = 0 \), it is clear from (M1) and \ref{mcompmorphism} that \( (g \ast f)(a) = 0 \) and \( (g \circ f)(a) = 0 \).
			
			Suppose \( a \neq 0 \).  By (D6), there is a \( c \in A \) such that \( c \dev a \).  By \ref{mdev}, \( f(c) \dev f(a) \) and thus \( (g \circ f)(c) \dev (g \circ f)(a) \).  By (D2), \( (g \circ f)(c) \leq (g \circ f)(a) \).  So \( (g \ast f)(a) = \bigvee_{c \dev a} (g \circ f)(c) \leq (g \circ f)(a)  \).
			
		\item
			Suppose \( a \dev b \).  By (D5), there is a \( c \in A \) such that \( a \dev c \dev b \).  By (D2), \( a \leq c \).  Applying \ref{mleq} twice, \( (g \circ f)(a) \leq (g \circ f)(c) \).  Hence \( (g \circ f)(a) \leq \bigvee_{d \dev b} (g \circ f)(d) = (g \ast f)(b) \).
			
		\item
			Suppose \( a \dev b \).  By \ref{mastleqcirc},
			\begin{align*}
				((h \ast g) \circ f)(a) &= (h \ast g)(f(a)) \\
				&\leq (h \circ g)(f(a)) \\
				&= (h \circ (g \circ f))(a)
			\end{align*}
			Since \( a \dev b \), by \ref{mdevcircleqast}, \( (g \circ f)(a) \leq (g \ast f)(b) \).  By \ref{mleq}, \( (h \circ (g \circ f))(a) \leq (h \circ (g \ast f))(b) \).
		\item
			Suppose \( a \dev b \).  By (D5), there is a \( c \in A \) such that \( a \dev c \dev b \).  By \ref{mastleqcirc}, \( (g \ast f)(a) \leq (g \circ f)(a) \).  By \ref{mleq},
			\begin{align*}
				(h \circ (g \ast f))(a) &\leq (h \circ (g \circ f))(a) \\
				&= (h \circ g)(f(a))
			\end{align*}	
			
			Since \( a \dev c \) then \( f(a) \dev f(c) \) by \ref{mdev}.  So by \ref{mdevcircleqast},
			\begin{align*}
				(h \circ g)(f(a)) &\leq (h \ast g)(f(c)) \\
				&= ((h \ast g) \circ f)(c)
			\end{align*}
			
			As \( c \dev b \), applying \ref{mdevcircleqast} again,
			\[ ((h \ast g) \circ f)(c) \leq ((h \ast g) \ast f)(b) \]
			
			Therefore, we have \( (h \circ (g \ast f))(a) \leq ((h \ast g) \ast f)(b) \).
	\end{enumerate}
\end{proof}

\begin{theorem}
	The class of de Vries algebras with de Vries morphisms and composition \( \ast \) as defined in \ref{mast} is a category.  We will denote this category as \( \DeV \).
\end{theorem}
\begin{proof}
	We only need to verify two things: that the composition of morphisms is associative and that each object has an identity morphism.

	First, we will show associativity of composition.  Let \( A,B,C,D \) be de Vries algebras and \( f:A \to B, g:B \to C, h:C \to D \) be de Vries morphisms.  We will show that \( h \ast (g \ast f) = (h \ast g) \ast f \).
	
	For \( a = 0 \), it is clear from \ref{mcompmorphism} and (M1) that \( (h \ast (g \ast f))(a) = 0 \) and \( ((h \ast g) \ast f)(a) = 0 \).
	
	Suppose \( a \neq 0 \).
	
	First we will show that \( (h \ast (g \ast f))(a) \leq ((h \ast g) \ast f)(a) \).

	By \ref{mdevcircleqastcomp}, for all \( b \in B \) such that \( b \dev a \),
	\[ (h \circ (g \ast f))(b) \leq ((h \ast g) \ast f)(a) \]
	
	So we have
	\begin{align*}
		(h \ast (g \ast f))(a) &= \bigvee_{b \dev a} (h \circ (g \ast f))(b)\\
		&\leq ((h \ast g) \ast f)(a)
	\end{align*}

	Now we will show that \( ((h \ast g) \ast f)(a) \leq (h \ast (g \ast f))(a) \).
	
	By (D6), there is a \( b \in A \) such that \( 0 \neq b \dev a \).  By (D5), there is a \( c \in A \) such that \( b \dev c \dev a \).  By \ref{mdevcompflip},
	\[ ((h \ast g) \circ f)(b) \leq (h \circ (g \ast f))(c) \]
	
	Since \( c \dev a \),
	\begin{align*}
		(h \circ(g \ast f))(c) &\leq \bigvee_{d \dev a} (h \circ (g \ast f))(d) \\
		&= (h \ast (g \ast f))(a)
	\end{align*}
	
	So we have, for all \( b \in A \) such that \( b \dev a \),
	\[ ((h \ast g) \circ f)(b) \leq (h \ast (g \ast f))(a) \]
	
	Therefore
	\begin{align*}
		((h \ast g) \ast f)(a) &= \bigvee_{b \dev a} ((h \ast g) \circ f)(b) \\
		&\leq (h \ast (g \ast f))(a)
	\end{align*}
	
	Now, let \( A,B \) be de Vries algebras and \( f \in \mor(A,B) \).  Define \( 1_A: A \to A \) as \( 1_A = \id_A \).  It is trivial to show that axioms (M1), (M2), and (M3) are satisfied.  That (M4) is satisfied follows directly from \ref{devveeunder}.  Therefore, \( 1_B \) is indeed a de Vries morphism.  Also, it is clear that \( 1_A \circ f = f = f \circ 1_B \).  Therefore, \( 1_A \) is an identity morphism.	
	
	For any \( a \in A \),
	\begin{align*}
		(f \ast 1_A)(a) &= \bigvee_{b \dev a} (f \circ 1_A)(b) \\
		&= \bigvee_{b \dev a} f(b)
	\end{align*}
	
	It follows from (M4) that
	\[ (f \ast 1_A)(a) = f(a) \]
	
	Similarly,
	\begin{align*}
		(1_B \ast f)(a) &= \bigvee_{b \dev a} (1_B \circ f)(b) \\
		&= \bigvee_{b \dev a} f(b) \\
		&= f(a)
	\end{align*}
		
	Therefore, \( 1_A \) is an identity morphism.
\end{proof}
