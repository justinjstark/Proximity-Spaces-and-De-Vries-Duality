\section{Categories}
\label{categories}

\begin{definition}
	A \defn{category} \( \fC \) consists of three entities:
	\begin{enumerate}
		\item A class of objects denoted \( \obj{\fC} \).
		\item A class of mappings between objects of \( \fC \) known as morphisms.  For \( A,B \in \obj{\fC} \), we denote the morphisms from \( A \) to \( B \) as \( \mor(A,B) \).
		\item A binary operation \( \circ \) such that for any \( A,B,C \in \obj{\fC} \), the operator \( \circ \) maps \( \mor(A,B) \times \mor(B,C) \) to \( \mor(A,C) \) and satisfies the following:
		\begin{enumerate}
			\item (Associativity) For any \( A,B,C,D \in \obj{\fC} \), \( f \in \mor(A,B) \), \( g \in \mor(B,C) \), and \( h \in \mor(C,D) \), \( h \circ ( g \circ f ) = (h \circ g) \circ f \).  In other words, the following commutes.
				\onecolumn{\xymatrix{
					A \ar[r]^{g \circ f} \ar@/_/[dr]_f \ar@{}[dr]^\circ & C \ar@/^/[dr]^h \ar@{}[dr]_\circ \\
					& B \ar[u]_g \ar[r]_{h \circ g} & D
				}}
			\item (Identity) For any \( A,B \in \obj{\fC} \) there is a morphism \( 1_A \in \mor(A,A) \) and a morphism \( 1_B \in \mor(B,B) \) such that for any \( f \in \mor(A,B) \), \( 1_B \circ f = f = f \circ 1_A \).
		\end{enumerate}
	\end{enumerate}
\end{definition}

Henceforth, when we say ``the category of \( A \) and \( B \)" we mean the category whose objects are the elements in \( A \) and whose morphisms between those objects are the elements in \( B \).  Unless otherwise stated, we will assume that composition is normal function composition.

\begin{examples}
	The following are examples of categories.
	\begin{enumerate}
		\item \( \SET \) The category of sets and functions.
		\item \( \CPTT \) The category of compact Hausdorff spaces and continuous functions.
		\item \( \BA \) The category of Boolean algebras and Boolean algebra homomorphisms.
		\item \( \TYCH \) The category of Tychonoff spaces and continuous functions.
		\item \( \ZDCPTT \) The category of zero dimensional compact Hausdorff spaces and continuous functions.
	\end{enumerate}
\end{examples}

\begin{definition}
	Let \( \fB \) and \( \fC \) be categories.  A \defn{covariant functor} \respdefn{contravariant functor} \( F \) is a correspondence from \( \fB \) to \( \fC \) such that
	\begin{enumerate}
		\item For each \( B \in \obj{\fB} \) there is a unique \( F(B) \in \obj{\fC} \)
		\item For \( A,B \in \obj{\fB} \) and \( f \in \mor(A,B) \), there is a unique \( F(f) \in \mor(F(A),F(B)) \) \sresp{\( F(f) \in \mor(F(B),F(A)) \)}.
		\item For all \( B \in \obj{\fB} \), \( F(1_{B}) = 1_{F(B)} \)
		\item For all \( A,B,C \in \obj{\fB} \), \( f \in \mor(A,B) \), and \( g \in \mor(B,C) \), \( F(g \circ f) = F(g) \circ F(f) \) \sresp{\( F(g \circ f) = F(f) \circ F(g) \)}.  In other words, the following respective diagram commutes.
				\threecolumn
				{
					\xymatrix
					{
						A \ar[r]^f \ar@/_/[dr]_{g \circ f} \ar@{}[dr]^\circ & B \ar[d]^g \\
						& C
					}
				}
				{
					\xymatrix
					{
						F(A) \ar[r]^{F(f)} \ar@/_/[dr]_{F(g \circ f)} \ar@{}[dr]^\circ & F(B) \ar[d]^{F(g)} \\
						& F(C)
					}
				}
				{
					{\respcolor
						\xymatrix
						{
							F(A) \ar@{}[dr]^\circ & F(B) \ar[l]_{F(f)} \\
							& F(C) \ar[u]_{F(g)} \ar@/^/[ul]^{F(g \circ f)}
						}
					}
				}
	\end{enumerate}
\end{definition}

\begin{definition}
	Let \( \fB \) and \( \fC \) be categories and \( F, G \) be covariant functors \resp{contravariant functors} from \( \fB \) to \( \fC \).  A correspondence \( \eta: F \to G \) is a \defn{natural transformation} if it maps each \( A \in \obj{\fB} \) to a unique \( \eta(A) \in \mor(F(A), G(A)) \) such that for any \( B \in \obj{\fB} \) and \( f \in \mor(A, B) \), \( \eta(B) \circ F(f) = G(f) \circ \eta(A) \) \sresp{\( \eta(A) \circ F(f) = G(f) \circ \eta(B) \)}.  In other words, the following respective diagram commutes.
\end{definition}

\threecolumn
{
	\xymatrix
	{
		A \ar[d]_f \\
		B
	}
}
{
	\xymatrix
	{
		F(A) \ar[r]^{\eta(A)} \ar[d]_{F(f)} \ar@{}[dr]|{\circ}	&	G(A) \ar[d]^{G(f)} \\
		F(B) \ar[r]_{\eta(B)} 																	& G(B)
	}
}
{
	{\respcolor
		\xymatrix
		{
			F(A) \ar[r]^{\eta(A)} \ar@{}[dr]|{\circ} 	& G(A)								\\
			F(B) \ar[r]_{\eta(B)} \ar[u]^{F(f)}				& G(B) \ar[u]_{G(f)}
		}
	}
}

\begin{example}
	Consider the category \( \TYCH \).  For \( X \in \obj{\TYCH} \), let \( \beta{X} \) be the Stone-Cech compactification of \( X \).  For \( X,Y \in \obj{\TYCH} \) and \( f \in \mor(X,Y) \), let \( \beta{f} \) be the continuous extension of \( f \) from \( \beta{X} \) to \( \beta{Y} \).  Let \( e(X): X \to \beta{X} \) be the inclusion map.  Then \( e: 1_{\TYCH} \to \beta \) is a natural transformation and we have the following commutative diagram.
	\onecolumn{\xymatrix{
		X \ar[r]^{e(X)} \ar[d]_{f} \ar@{}[dr]|\circ	& \beta{X} \ar[d]^{\beta(f)} \\
		Y \ar[r]_{e(Y)}							& \beta{Y}
	}}
\end{example}

\begin{definition}
	Let \( \fC \) be a category and \( A,B \in \obj{\fC} \).  The morphism \( f \in \mor(A,B) \) is an \defn{isomorphism} if there is a \( g \in \mor(B,A) \) such that \( f \circ g = 1_B \) and \( g \circ f = 1_A \).
\end{definition}

\begin{definition}
	Let \( \fB \) and \( \fC \) be categories and \( F, G \) be functors (either covariant or contravariant) from \( \fB \) to \( \fC \).  A natural transformation \( \eta:F \to G \) is a \defn{natural isomorphism} if for all \( B \in \obj{\fB} \), \( \eta(B) \) is an isomorphism.
\end{definition}

If a natural isomorphism exists between two categories \( \fB \) and \( \fC \), we write \( \fB \cong \fC \).

\begin{definition}
	Two categories \( \fB \) and \( \fC \) are said to be \defn{equivalent} \respdefn{dually equivalent} if there exist covariant functors \sresp{contravariant functors} \( F \) from \( \fB \) to \( \fC \) and \( G \) from \( \fC \) to \( \fB \) as well as natural isomorphisms \( \eta: F \circ G \to 1_\fC \) and \( \zeta: 1_\fB \to G \circ F \).
\end{definition}

When two categories are dually equivalent, we will often say the two categories are dual or have duality.

%The following example which is a result of Marshall Stone \cite{stone}.

\begin{example}[Stone Duality \cite{stone}]
	For \( B \in \obj{\BA} \), let \( S(B) \) denote the Stone space of \( B \).  For \( A,B \in \obj{\BA} \) and \( f \in \mor(A,B) \), define \( S(f):S(B) \to S(A) \) by \( S(f)(\cU) = \{ a \in A: f(a) \in \cU \} \).  The mapping \( S \) is a contravariant functor from \( \BA \) to \( \ZDCPTT \).
	
	For \( X \in \obj{\ZDCPTT} \), the clopen sets of \( X \), \( \Cl(X) \), form a Boolean algebra.  For \( X,Y \in \obj{\ZDCPTT} \) and \( f \in \mor(X,Y) \), define \( \Cl(f): \Cl(Y) \to \Cl(X) \) by \( \Cl(f)(U) = f^\leftarrow[U] \).  The mapping \( \Cl \) is a contravariant functor from \( \ZDCPTT \) to \( \BA \).
	
	Define \( \eta:A \to \Cl(S(A)) \) by \( \eta_A(a) = \{ \cU \in S(A): a \in \cU \} \) and \( \zeta_X:S(\Cl(X)) \to X \) by \( \zeta_X(\cU) = \bigcap{\cU} \).
	
	For any \( A \in \obj{\BA} \), \( \eta_A \) is a Boolean isomorphism and for any \( X \in \obj{\ZDCPTT} \), \( \zeta_X \) is a homeomorphism. 
	
	For \( A,B \in \obj{\BA} \); \( f \in \mor(A,B) \); \( X,Y \in \obj{\ZDCPTT} \); and \( g \in \mor(X,Y) \), we have the following diagram.
	
	\twocolumn
	{
		\xymatrix
		{
			A \ar[r]^{\eta_A} \ar[d]_f \ar@{}[dr]|{\circ}	&	**[r] \Cl(S(A)) \ar[d]^{\Cl(S(f))} \\
			B \ar[r]_{\eta_B} 														& **[r] \Cl(S(B))
		}
	}
	{
		\xymatrix
		{
			**[l] S(\Cl(X)) \ar[r]^{\zeta_X} \ar[d]_{S(\Cl(g))} \ar@{}[dr]|{\circ}	&	X \ar[d]^g \\
			**[l] S(\Cl(Y)) \ar[r]_{\zeta_Y} 																				& Y
		}
	}
	
	These diagrams commute so \( \eta \) and \( \zeta \) are natural isomorphisms.  Therefore, \( \BA \) and \( \ZDCPTT \) are dually equivalent.
\end{example}
